\section{Hardware Abstraction Layer (HAL)}
\begin{itemize}
  \item In Programmen für Embedded Systems gibt es sehr häufig Zugriffe auf Hardware-Register, d.h. die Programmierung erfolgt häufig durch Setzen oder Löschen, bzw. Lesen einzelner Bits
  \item Wenn diese Zugriffe direkt bei Bedarf getätigt werden, entstehen sehr unleserliche und auch fehlerträchtige Programme
  \item Wenn die Zielhardware geändert wird, muss praktisch die ganze Software neu geschrieben werden, d.h. die Portierbarkeit ist sehr schlecht
  \item \textbf{Diese Registerzugriffe sollen unbedingt in einem HAL abstrahiert werden}
  \begin{itemize}
    \item Wenn es richtig gemacht wird, ist es nicht mit einem enormen Overhead verbunden, wie oft behauptet wird.
    \item And: Macro-based HALs suffer from all the macro drawbacks
  \end{itemize}
\end{itemize}

\subsection{Organisation des HALs}
\begin{itemize}
  \item Unterteilung in
  \begin{itemize}
    \item Board Support Library (BSL)
    \item Chip Support Library (CSL)
  \end{itemize}
  \item Die CSL abstrahiert den Chip, d.h. den Mikrocontroller und wird häufig vom $\mu$C-Hersteller zur Verfügung gestellt.
  \item Die BSL abstrahiert das PCB und muss vom Hersteller des Boards zur Verfügung gestellt werden.
\end{itemize}

\subsection{HAL in C}
Client Program in C
\lstinputlisting[style=C]{snippets/HAL/client.c}
\begin{itemize}
  \item Ab C99 wird auch Inlining unterstützt
  \item Die Inline-Funktionen müssen in Headerfiles definiert sein, damit der Compiler auch Inlining machen kann
  \item Wenn die Funktionen \lstinline[style=C]{static} deklariert werden, wird garantiert, dass die Funktionen nicht auch noch im Objectfile als Funktion vorhanden sind
  \item Damit die Namen eindeutig sind, sollen die Unitkürzel \textbf{csl}, bzw. \textbf{bsl} verwendet werden (in C++ ist das dank Namespaces viel eleganter und einfacher)
\end{itemize}
\begin{lstlisting}[style=C]
enum {csl_bit0 = 0x00000001, csl_bit1 = 0x00000002, csl_bit2 = 0x00000004, csl_bit3 = 0x00000008,
    csl_bit4 = 0x00000010, csl_bit5 = 0x00000020, csl_bit6 = 0x00000040, csl_bit7 = 0x00000080,
    // ...
    csl_bit28 = 0x10000000, csl_bit29 = 0x20000000, csl_bit30 = 0x40000000, csl_bit31 = 0x80000000};
\end{lstlisting}
\begin{lstlisting}[style=C]
static inline void csl_hwReg32SetBits(volatile uint32_t* reg, uint32_t bits) {*reg |= bits;}
static inline void csl_hwReg16ClearBits(volatile uint16_t* reg, uint16_t bits) {*reg &= ~bits;}
\end{lstlisting}

\subsubsection{CSL in C}
\textbf{Port definitions}
\lstinputlisting[style=C]{snippets/HAL/cslPort.h}

\textbf{Pin definitions}
\lstinputlisting[style=C]{snippets/HAL/cslPin.h}

\subsubsection{BSL in C}
In der Board Support Library wird die Verbindung der effektiv auf dem Board vorhandenen Komponenten mit dem $\mu$C über die CSL abstrahiert.
\lstinputlisting[style=C]{snippets/HAL/bsl.h}

\subsection{HAL in C++}
Client Program in C++
\lstinputlisting[style=Cpp]{snippets/HAL/client.cpp}
\begin{itemize}
  \item Die Inline-Funktionen müssen in Headerfiles definiert sein, damit der Compiler auch Inlining machen kann
  \item Damit die Namen eindeutig sind, sollen die Namespaces \textbf{csl}, bzw. \textbf{bsl} definiert werden.
\end{itemize}
\begin{lstlisting}[style=Cpp]
namespace csl
{
enum {bit0 = 0x00000001, bit1 = 0x00000002, bit2 = 0x00000004, bit3 = 0x00000008,
    bit4 = 0x00000010, bit5 = 0x00000020, bit6 = 0x00000040, bit7 = 0x00000080,
    // ...
    bit28 = 0x10000000, bit29 = 0x20000000, bit30 = 0x40000000, bit31 = 0x80000000};

template<typename RegType>
class HwReg{
  static void setBits(volatile RegType& reg, const RegType& bits) {reg |= bits;}
  static void clearBits(volatile RegType& reg, const RegType& bits) {reg &= ~bits;}
};
}
\end{lstlisting}

\subsubsection{CSL in C++}
\textbf{Port definitions}
\lstinputlisting[style=Cpp]{snippets/HAL/cslPortCpp.h}
\textbf{Pin definitions}
\lstinputlisting[style=Cpp]{snippets/HAL/cslPinCpp.h}

\subsubsection{BSL in C++}
In der Board Support Library wird die Verbindung der effektiv auf dem Board vorhandenen Komponenten mit dem $\mu$C über die CSL abstrahiert.
\lstinputlisting[style=Cpp]{snippets/HAL/bslCpp.h}
