\section{Bootstrap}
\subsection{Programmdownload \buch{p.3}}
\begin{itemize}
	\item Während der Entwicklung ist das Embedded System häufig über eine JTAG-Verbindung mit der 
	      Entwicklungsumgebung verbunden 
	\item JTAG-Schnittstellen erfordern häufig (teure) Adapter, die für die Entwicklung akzeptabel, 
	      im Betrieb aber zu teuer sind
	\item Für den Programmdownload im Betrieb steht häufig nur eine serielle Schnittstelle zur Verfügung
	\item Der Bootloader ist für die erste Initialisierung zuständig und für das Starten des eigentlichen
	      Programms (OS)
  \item Der Bootloader muss immer auf einem bootfähigen und nichtflüchtigen Medium gespeichert werden.
\end{itemize}
\subsection{Startvorgang (Booting) \buch{p.4}}
\begin{itemize}
	\item Typischer Bootvorgang
	\begin{enumerate}
		\item Beim Aufstarten wird immer zuerst das
		Bootloader-Programm gestartet
		\item In diesem Programm wird eine Weile
		gewartet, ob beispielsweise auf der RS-
		232 ein neues Programm in den
		Programmbereich geschrieben werden
		soll
		\item Wenn ein neues Programm geschrieben
		werden soll, dann wird dieses mit Hilfe
		eines definierten Protokolls übertragen
		und geschrieben
		\item Wenn das Programm fertig übertragen
		wurde oder das bestehende ausgeführt
		werden soll, wird mit der Ausführung an
		dieser Stelle fortgefahren
	\end{enumerate}
	\item Um die sichere Übertragung zu gewährleisten, wird diese meist mittels CRC gesichert, bei jedem Datenblock wird zusätzlich eine Prüfsumme mitgeschickt
	\item Teilweise wird die Übertragung auch verschlüsselt, damit eine bösartige Änderung des Programms verhindert werden kann
	\item Häufig wird der Bootloader in mehreren Stufen aufgebaut. z.B.
	\begin{itemize}
		\item Erste Stufe initialisiert die Hardware und lädt ein bestimmtes Filesystem
		\item Die nächste Stufe ist als File in diesem Filesystem gespeichert und startet beispielsweise das Betriebssystem
	\end{itemize}
	\item Vorteile:
	\begin{itemize}
		\item Die erste Stufe ist nur hardwarespezifisch, jedoch völlig unabhängig von Filesystem und Betriebssystem
		\item Dadurch können diese Teile beliebig ausgetauscht und geändert werden
	\end{itemize}
\end{itemize}

