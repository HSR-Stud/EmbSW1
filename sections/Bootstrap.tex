\section{Bootstrap}
\subsection{Programmdownload \buch{p.3}}
\begin{itemize}
	\item Während der Entwicklung ist das Embedded System häufig über eine JTAG-Verbindung mit der Entwicklungsumgebung verbunden 
	\item JTAG-Schnittstellen erfordern häufig (teure) Adapter, die für die Entwicklung akzeptabel, im Betrieb aber zu teuer sind
	\item Für den Programmdownload im Betrieb steht häufig nur eine serielle Schnittstelle zur Verfügung
\end{itemize}
\subsection{Startvorgang (Booting) \buch{p.4}}
\begin{itemize}
	\item Typischer Bootvorgang
	\begin{enumerate}
		\item Beim 
	\end{enumerate}
	\item um die sichere Übertragung zu gewährleisten, wird diese meist mittels CRC gesichert, bei jedem Datenblock wird zusätzlich eine Prüfsumme mitgeschickt
	\item Teilweise wird die Übertragung auch verschlüsselt, damit eine bösartige Änderung des Programms verhindert werden kann
	\item Häufig wird der Bootloader in mehreren Stufen agebaut,z.B.
	\begin{itemize}
		\item erste Stufe initialisiert die Hardware und lädt ein bestimmtes Filesystem
		\item die nächste Stufe ist als File in diesem Filesystem gespeichert und startet beispielsweise das Betriebssystem
	\end{itemize}
	\item Vorteile:
	\begin{itemize}
		\item die erste Stufe ist nur hardwarespezifisch,jedoch völlig unabhängig von Filesystem und Betriebssystem
		\item dadurch können diese Teile beliebig ausgetauscht und geändert werden
	\end{itemize}
\end{itemize}

