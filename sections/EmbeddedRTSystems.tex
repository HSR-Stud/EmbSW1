\section{Embedded RT Systems \buch{p.1}}
\subsection{Charakterisierung Embedded Systems \buch{p.3}}
Ein Embedded System \ldots
\begin{itemize}
  \item \ldots ist ein System, das einen Computer beinhaltet, aber keiner ist
  \item \ldots besteht üblicherweise aus HW und SW
  \item \ldots ist häufig ein Control System
Charakterisierung von Embedded Systems
  \begin{itemize}
    \item \textbf{Reaktive Systeme} : Interagieren mit ihrer Umgebung
    \item \textbf{Echtzeitsysteme/Real-time systems} : Definierbare zeitliche
    Anforderungen erfüllen
    \item \textbf{Verlässliche Systeme/Dependable systems}: Sehr hohe
    Zuverlässigkeitsanforderungen erfüllen
    \item \textbf{Weitere Anforderungen/Charakteristiken}: 
    \begin{itemize}
      \item kleiner Energieverbrauch
      \item kleine physikalische Abmessungen
      \item Lärm, Vibration, Feuchtigkeit etc\ldots
    \end{itemize}
  \end{itemize}
\end{itemize}

Verfügbarkeit: Anteil der Betriebsdauer, in  der das System seine Funktion
erfüllt: 

\begin{equation}
	\hat{f}(x,y)= \left[\prod\limits_{(s,t)\in S_{xy}} g(s,t)\right]^\frac{1}{mn}
\end{equation}