\subsection{Delegates and Events}
\subsubsection{Declaration of a delegate}
\begin{lstlisting}
delegate void Notifier (string sender);
\end{lstlisting}

\subsubsection{Delegate variable}
Declaraciton
\begin{lstlisting}
Notifier greetings;
\end{lstlisting}
\begin{itemize}
  \item A delegate variable can have the value "`null"'
  \item If "`null"', a delegate must not be called (exception)
  \item Delegate variables are first class objects, can be stored in data
  structure, passed as a parameter etc.
\end{itemize}

\subsubsection{Assigning a method}
\begin{lstlisting}
void SayHello(string sender) {
	Console.WriteLine("Hello from" + sender);
}
greetings = new Notifier(SayHello);
//since C# 2.0 also possible:
greetings = SayHello;
\end{lstlisting}
Every matching method can be assigned to a delegate variable.
\begin{lstlisting}
void SayGoodBye(string sender) {
	Console.WriteLine("Good bye from" + sender);
}
greetings = SayGoodBye;

greetings("John"); //console: "Good bye from John"
\end{lstlisting}

\subsubsection{Calling a delegate variable}
\begin{lstlisting}
greetings("John");
\end{lstlisting}

\subsubsection{Multicast Delegates}
\begin{lstlisting}
Notifier greetings
greetings = SayHello;
greetings += SayGoodBye;
greetings("John");
// "Hello from John"
// "Good bye from John"
\end{lstlisting}

\subsubsection{Event}