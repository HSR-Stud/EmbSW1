\subsection{Delegates and Events}
\subsubsection{Declaration of a delegate}
\begin{lstlisting}
delegate void Notifier (string sender);
\end{lstlisting}

\subsubsection{Delegate variable}
Declaraciton
\begin{lstlisting}
Notifier greetings;
\end{lstlisting}
\begin{itemize}
  \item A delegate variable can have the value "`null"'
  \item If "`null"', a delegate must not be called (exception)
  \item Delegate variables are first class objects, can be stored in data
  structure, passed as a parameter etc.
\end{itemize}

\subsubsection{Assigning a method}
\begin{lstlisting}
void SayHello(string sender) {
	Console.WriteLine("Hello from" + sender);
}
greetings = new Notifier(SayHello);
//since C# 2.0 also possible:
greetings = SayHello;
\end{lstlisting}
Every matching method can be assigned to a delegate variable.
\begin{lstlisting}
void SayGoodBye(string sender) {
	Console.WriteLine("Good bye from" + sender);
}
greetings = SayGoodBye;

greetings("John"); //console: "Good bye from John"
\end{lstlisting}

\subsubsection{Calling a delegate variable}
\begin{lstlisting}
greetings("John");
\end{lstlisting}

\subsubsection{Multicast Delegates}
\begin{lstlisting}
Notifier greetings
greetings = SayHello;  //reine Zuweisung (alle anderen Funktionen werden wieder geloescht)
greetings += SayGoodBye;  //Funktion hinzufuegen
greetings("John");
// "Hello from John"
// "Good bye from John"
greetings -= SayHello;  //Funktion entfernen
\end{lstlisting}

\begin{itemize}
  \item  Eine delegate Variable beinhaltet Funktionen und ihre Empfänger, aber nicht ihre Parameter!\newline
         \textit{new Notifier(myObj.SayHello);}
  \item  \textit{obj} kann \textit{this} sein \newline
         \textit{new Notifier(SayHello);}
  \item  Eine Funktion kann statisch sein. In diesem Fall muss der Klassen spezifiziert werden. \newline
         \textit{new Notifier(MyClass.StaticSayHello);}
  \item  Eine Funktion darf nicht abstrakt sein, aber virtual, override und new sind erlaubt.
  \item  Die Signatur der Funktion muss mit der Signatur vom Delegate Typ übereinstimmen:
			   \begin{itemize}
			     \item gleiche Anzahl von Parameter
			     \item gleiche Parametertypen
			     \item gleiche Parameter Art (value, ref/out)
			   \end{itemize}
\end{itemize}

\subsubsection{Event}
An event is a special delegate field. Reasons for events:
\begin{itemize}
  \item Only the class that declares the event can fire it
  \item Other classes may change event fields only with += or -=
\end{itemize}

\textbf{Beispiel}
\begin{lstlisting}[style=Csharp]
	class Model {
	  public event Notifier notifyViews;
	  public void Change() { ... notifyViews("Model"); }
	}
	
	class View {
  	public View(Model m) { m.notifyViews += Update; }
	  void Update(string sender) { Console.WriteLine(sender + " was changed"); }
	}
	
	class Test {
  	static void Main() {
    	Model model = new Model();
	    new View(model); new View(model); ...
	    model.Change();
	  }
	}
\end{lstlisting}

\subsubsection{Declaration of an event}
\begin{lstlisting}
delegate void SomeEvent(object source, SomeEventArgs e);
\end{lstlisting}
1. Parameter: Sender of the event (type object)
2. Parameter: event info (subclass of System.EventArgs)


\subsubsection{Einfaches erstellen von Delegates}
\begin{lstlisting}[style=CSharp]
	delegate void Printer(string s);
	void Foo(string s) {
	  Console.WriteLine(s);
	}
	Printer print;
	print = new Printer(this.Foo);
	print = this.Foo;
	print = Foo;
	
	delegate double Function(double x);
	double Foo(double x) {
	  return x * x;
	}
	Printer print = Foo;    //assigns Foo(string s)
	Function square = Foo;  //assigns Foo(double x)
\end{lstlisting}


\newpage
\subsubsection{Ordinary Delegates}
\begin{multicols}{2}
	
\begin{lstlisting}[style=CSharp]
delegate void Visitor(Node p);
class List {
  Node[] data = ...;
  ...
  public void ForAll(Visitor visit) {
  for (int i = 0; i < data.Length; i++)
  visit(data[i]);
}
\end{lstlisting}

\columnbreak
  
\begin{lstlisting}[style=CSharp]
class C {
  int sum = 0;
  void SumUp(Node p) { sum += p.value; }
  void Print(Node p) { Console.WriteLine(p.value); }
  void Foo() {
    List list = new List();
    list.ForAll(SumUp);
    list.ForAll(Print);
  }
}
  \end{lstlisting}

\end{multicols}

\subsubsection{Anonymus Methods}
\begin{lstlisting}[style=CSharp]
delegate void Visitor(Node p);
class C {
  void Foo() {
    List list = new List();
    int sum = 0;
    list.ForAll(delegate (Node p) { Console.WriteLine(p.value); });
    list.ForAll(delegate (Node p) { sum += p.value; });
  }
}

class List {
  ...
  public void ForAll(Visitor visit) {
  ...
  }
}
\end{lstlisting}
\begin{itemize}
  \item  Der Funktions Body (\textit{sum += p.value;}) ist an der Stelle definiert.
  \item  Es braucht keine Funktionsdeklaration
  \item  Die Anynume Funktion hat Zugriff auf \textit{Foo's} lokale Variable \textit{sum}
  \item  Im oberen Beispiel ist die Anonyme Funktion vom Typ Visitor
  \item  Anonyme Funktionen dürfen keine formalen Parameter vom Typ \textit{params T[]} haben.
  \item  Anonyme Funktionen haben keinen Zugriff auf \textit{ref} oder \textit{out}
         Parameter der umschliessenden Funktion.
\end{itemize}

\subsubsection{Komplettes Beispiel}
\lstinputlisting{snippets/Main.cs}
