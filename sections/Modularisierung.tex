\section{Modularisierung}
Modularisierung ist die Aufteilung eines Ganzen in Teile, die als Module oder Komponenten bezeichnet werden. Bei geeigneter Form und Funktion können sie zusammengefügt werden oder über entsprechende Schnittstellen interagieren.

\subsection{Motivation}
Bei Kleinstsoftware braucht es kein systematisches Design, da man diese meist auch nicht ausbauen möchte. Vernünftige Softwaresysteme benötigen einen systematischen Designansatz mit einem strukturierten Aufbau.\\\\
Wichtig ist daher:
	\begin{itemize}
	\item Entwicklung im Team vs. One-person-show
  	\item Schnittstellen definieren
	\end{itemize}

\subsection{Grundprinzip}
\begin{itemize}
  \item Divide et impera (Teile und herrsche)
  \begin{itemize}
    \item Problem aufteilen in Unterprobleme, diese einzeln angehen
    \item Zuerst Grobentwurf, dann verfeinern (Top-Down-Prinzip)
  \end{itemize}
  \item \textbf{Ziel:} Reduktion der Komplexität
\end{itemize}

\subsection{Information Hiding}
\begin{itemize}
  \item Nutzer des Moduls braucht nur die Schnittstelle zu kennen (*.h files)
  \item Über inneren Aufbau dürfen keine Annahmen getroffen werden
  \item Der innere Aufbau (*.cpp files) muss nicht bekannt sein
\end{itemize}

\subsection{Hauptkriterien der Zerlegung}
\begin{itemize}
  \item Kopplung/Coupling
  \begin{itemize}
    \item Mass für Komplexität der Schnittstelle
  \end{itemize}
  \item Kohäsion/Cohesion
  \begin{itemize}
    \item Aussage wie stark eine funktionale Einheit wirklich zusammengehört
    \item Mass für die Stärke des inneren Zusammenhangs 
  \end{itemize}
\end{itemize}

\textbf{Ziel:} Schwache Kopplung, starke Kohäsion 

\begin{minipage}{9cm}
  \includegraphics[width=8cm]{images/Modularisierung/BeispieleKopplungKohaesion.png}
\end{minipage}
\begin{minipage}{8cm}
  \begin{itemize}
    \item Bild a: Schwache Kopplung, starke Kohäsion (gut)
    \item Bild b: Starke Kopplung, schwache Kohäsion (schlecht)
  \end{itemize}
\end{minipage}

\begin{multicols}{2}
\subsection{Definitionen Kopplung}
\begin{itemize}
  \item Keine direkte Kopplung (Schwache Kopplung $\rightarrow$ \textbf{GUT})
  \item Datenkopplung
  \begin{itemize}
    \item Kommunikation ausschliesslich über Parameter 
  \end{itemize}
  \item Datenbereichskopplung
   \begin{itemize}
    \item Modul hat Zugriff auf Datenstruktur eines anderen Moduls
    \item Es werden nur einzelne Komponenten benötigt
  \end{itemize}
  \item Steuerflusskopplung (Control Flow)
  \begin{itemize}
    \item Modul beeinflusst Steuerfluss eines anderen Moduls
  \end{itemize}
  \item Globale Kopplung
  \begin{itemize}
    \item Kommunikation über globale Variabeln
    \item Jedes Modul hat Zugriff
  \end{itemize}
  \item Inhaltskopplung (Starke Kopplung $\rightarrow$ \textbf{SCHLECHT})
  \begin{itemize}
    \item Aus einem Modul werden lokale Daten eines anderen Moduls modifiziert,
    obwohl dieses Modul NICHT vom anderen Modul aufgerufen worden ist. 
    \item \textbf{Todsünde}
  \end{itemize}
\end{itemize}
\columnbreak
\subsection{Definitionen Kohäsion}
\begin{itemize}
  \item Funktionale Kohäsion (Starke Kohäsion $\rightarrow$ \textbf{GUT})
  \begin{itemize}
    \item Teile einer Einheit bilden zusammen eine Funktion 
  \end{itemize}
  \item Sequentielle Kohäsion
   \begin{itemize}
    \item Teilfunktionen einer Einheit werden nacheinander ausgeführt
    \item Das Ergebnis einer Teilfunktion ist die Eingabe der nächsten
  \end{itemize}
  \item Kommunikative Kohäsion
  \begin{itemize}
    \item Teilfunktionen einer Einheit werden auf den gleichen Daten ausgeführt
    \item Reihenfolge spielt keine Rolle
  \end{itemize}
  \item Prozedurale Kohäsion
  \begin{itemize}
    \item Teilfunktionen werden nacheinander ausgeführt
    \item Sind über Steuerfluss verknüpft
  \end{itemize}
  \item Zeitliche Kohäsion
  \begin{itemize}
    \item Teile einer Einheit sind alle zu einer bestimmten auszuführen
  \end{itemize}
  \item Logische Kohäsion
  \begin{itemize}
    \item Teilfunktionen einer Einheit gehören zu einer Klasse
  \end{itemize}
  \item Zufällige Kohäsion (Schwache Kohäsion $\rightarrow$ \textbf{SCHLECHT})
  \begin{itemize}
    \item Teilfunktionen einer Einheit haben keinen sinnvollen Zusammenhang
  \end{itemize}
\end{itemize}
\end{multicols}
\begin{minipage}{9cm}
\subsubsection{Ziele bezüglich Kohäsion}
  \begin{itemize}
  \item Kohäsion maximieren
  \item Idealerweise führt eine Unit nur eine Aufgabe (Funktion) aus
  \item Starke Kohäsion steigert
    \begin{itemize}
    \item Wartbarkeit
    \item Änderbarkeit
    \item Verständlichkeit
    \end{itemize}
  \item \textbf{Zusammengehörendes zusammen nehmen!}
  \item \textbf{Starke Kohäsion führt zu schwacher Kopplung, aber nicht umgekehrt!!}
  \end{itemize}
\end{minipage}
\hfill
\begin{minipage}{9cm}
  \subsubsection{Bestimmung der Kohäsion}
  \includegraphics[width=9cm]{images/Modularisierung/Kohaesionsbestimmung.png}
\end{minipage}

\subsection{Schlechte vs. gute Modularisierung}
\begin{minipage}[t]{10cm}
\textbf{Beispiel schlechter Modularisierung:}\\
\includegraphics[width=9cm]{images/Modularisierung/SchlechtesBeispielModularisierung.png}
\end{minipage}
\begin{minipage}[t]{8cm}
\textbf{Beispiel guter Modularisierung:}\\
\includegraphics[width=8cm]{images/Modularisierung/GutesBeispielModularisierung.png}
\end{minipage}
