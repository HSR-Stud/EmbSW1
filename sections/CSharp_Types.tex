\subsection{Types}

In C$\sharp$ git es drei Kategorien von Typen. 
\begin{itemize}
	\item Value Types
	\begin{itemize}
		\item Primitive Types (bool, int, char, long \ldots)
		\item Enums
		\item Structs
	\end{itemize}
	\item Reference Types
	\begin{itemize}
		\item Class
		\item Interface
		\item Arrays
		\item Delegates
	\end{itemize}
	\item Pointers
\end{itemize}

Unter \textbf{User-defined Types} versteht man Enums, Structs, Classes, Interface, Arrays and Delegates.
Alle \textsl{Types} sind kompatiebel mit \textsl{objetct}

\subsubsection{Value vs. Reference}
\begin{tabular}{|c|c|c|}
	\hline
	\textbf{Variable of} & \textbf{Values Types}  & \textbf{Reference Types} \\ \hline
	     constrins       &         value          &         referenc         \\ \hline
	     stored on       & stack (or in a object) &           heap           \\ \hline
	   initialization    &        0, false        &           null           \\ \hline
	     assignment      &    copies the value    &   copies the reference   \\ \hline
\end{tabular} 
\subsubsection{Kompatibilität}
Eine vollwertige Kompatibilität besteht nur, wenn von einem kleinerem Type auf einen grösseren Type gewechselt wird.

\begin{figure}[h]
	\centering
	\includegraphics[height=3cm, ]{images/CSharp/TypenKompatibilitaet}
	\caption{Typen Kompatibilität}
\end{figure}

folgende Beispiel sind gültig:\\

intVar = shortVar;\\
intVar = charVar;\\
floatVar = charVar;\\
decimalVar = (decimal)doubleVar;\\

\subsubsection{Enumaration}
\begin{lstlisting}
	enum Color {Red, Blue, Green} // values: 0, 1, 2
	enum Access {Personal=1, Group=2, All=4}
	enum Access1 : byte {Personal=1, Group=2, All=4}
\end{lstlisting}

\subsubsection{Arrays}
Wie in in der Programiersprache C, beginnt der Index auch hier bei der Stelle 0.\\ \\
\textbf{One-dimensinal arrays}
\begin{lstlisting}
	int[] a = new int[3];
	int[] b = new int[] {3, 4, 5};
	int[] c = {3, 4, 5};
	SomeClass[] d = new SomeClass[10]; // array of references
	SomeStruct[] e = new SomeStruct[10]; // array of values (directly in the array)
\end{lstlisting}

\textbf{Multi-dimensinal arrays (jagged)}\\
Hier können die Zeileneinträge verschieden lang sein, das heiss die Matrize ist nicht Rechteckig!
\begin{lstlisting}
	int[][] a = new int[2][]; // array of references to other arrays
	a[0] = new int[] {1, 2, 3}; // cannot be initialized directly
	a[1] = new int[] {4, 5, 6};
	
	int x = a[0][1];
\end{lstlisting}

\textbf{Multi-dimensinal arrays (rectangular)}\\
Hier sind die Einträge in jeder Zeile gleich lang, dadurch wird die Matrize Rechteckig!
\begin{lstlisting}
	int[,] a = new int[2, 3]; // block matrix 
	int[,] b = {{1, 2, 3}, {4, 5, 6}}; // can be initialized directly 
	int[,,] c = new int[2, 4, 2];
	
	int x = a[0,1];
\end{lstlisting}

\textbf{Strings)}\\
\begin{itemize}
	\item Können mit "+" verbunden werend z.B. "Don" + s
	\item Könne selektiert werden: z.B. s[i]
	\item Strings sind vom Type reference
	\item Können mit == und != verglichen werden
	\item und vieles mehr \ldots
\end{itemize}

\subsubsection{Variable-length Array}
\begin{multicols}{2}

	\begin{lstlisting}
		using System;
		using System.Collections;
		class Test {
			static void Main() {
			ArrayList a = new ArrayList();
			a.Add("Charly");
			a.Add("Delta");
			a.Add("Alpha");
			a.Sort();
			for (int i = 0; i < a.Count; i++)
				Console.WriteLine(a[i]);
			}
		};
	\end{lstlisting}
	\columnbreak
	
	\textbf{Output}\\
		Alpha\\
		Charly\\
		Delta\\
\end{multicols}


\subsubsection{Associative Array}
\begin{multicols}{2}

	\begin{lstlisting}
		using System;
		using System.Collections;
		class Test {
			static void Main() {
			Hashtable phone = new Hashtable();
			phone["Karin"] = 7131;
			phone["Peter"] = 7130;
			phone["Wolfgang"] = 7132;
			foreach (string key in phone.Keys)
				Console.WriteLine("{0} = {1}", key, phone[key]);
			}
		};
	\end{lstlisting}
	\columnbreak
	
	\textbf{Output}\\
		Karin = 7131\\
		Peter = 7130\\
		Wolfgang = 7132\\
\end{multicols}


\subsubsection{Classes vs. Structs}
\begin{tabular}{|c|c|}
	\hline
	           \textbf{Classes }            &               \textbf{Structs}               \\ \hline
	            Reference types             &                 Value types                  \\ \hline
	          support inheriance            &                no inheritance                \\ \hline
	       can implements interfaces        &          can implements interfaces           \\ \hline
	may declare a parameterless constructor & must not declare a parameterless constructor \\ \hline
	         may have a destructor          &                no destructor                 \\ \hline
\end{tabular} 

\subsubsection{Boxing Uniboxing}
\begin{figure}[h]
	\centering
	\includegraphics[height=3cm, ]{images/CSharp/BoxingUnboxing}
	\caption{Boxing Unboxing}	
\end{figure}

\textbf{Boxing}\\
Boxing ist wenn man eine Variable vom Stack auf den Heap legt. Das heisst man macht aus einer \textit{Value Type} einen \textit{Reference Type}.\\ 

\textbf{Unboxing}\\
Unboxing ist genau das gegenteil. Hier holt man eine Variable vom Heap und legt sie auf den Stack. Das heisst man macht aus einem \textit{Reference Type} einen \textit{Value Type}.