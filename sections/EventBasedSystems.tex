\section{Event Based Systems \buch{p.1}}
\begin{lstlisting}

\end{lstlisting}
\subsection{Reaktive Systeme \buch{p.3}}
Ein Reaktives System befindet sich in ständiger Interaktion mit der Umgebung, wobei die Umgebung dominiert und das System sich dieser unterordnet.
\begin{itemize}
	\item Viele Embedded Systems sind auch reaktive Systeme
	\item Reaktive Systeme reagieren auf (oft externe) Ereignisse
	\begin{itemize}
	\item Digitale Inputs
	\item Erreichen von Analogen Schwellwerten
	\item Timer
	\item Buttonclicks und ähnliches in GUI
	\item etc.
	\end{itemize}
\end{itemize}
\subsection{Umsetzung von Ereignissen \buch{p.4-10}}
\begin{itemize}
	\item Ereignisse sind per Definition asynchron
	\item Ein "normales" Programm ist immer synchron(zuerst das, dann das, dann...)
	\item Polling
	\begin{itemize}
		\item[+] einfach zu implementiern
		\item sehr viele Leerabfragen $\rightarrow$ unnötige Prozessorbelastung
		\item Leerabfragen können durch periodisches Abfragen (Kopplung an Timer) reduziert aber nicht vermieden werden.
		\item Die maximale Reaktionszeit wird durch die Abfrageperiode und die Anzahl Abfragen definiert
	\end{itemize}
	\item Interrupt
	\begin{itemize}
		\item Non-vectored Interrupt(zentral)
		\begin{itemize}
			\item Alle Interrupts verzweigen zu einer gemeinsamen Adresse. Dort wird die Ursache bestimmt und zu einer spezifischen Behandlungsroutine verzweigt.
		\end{itemize}
		\item Vectored Interrupt(spezifisch)
		\begin{itemize}
			\item In einer Tabelle(Interruptvektortabelle) wird gespeichert, wohin bei welchem Interruptvektor verzweigt werden muss.
		\end{itemize}
	\end{itemize}
	\item	Interruptvektortabelle(IVT)
	\begin{itemize}
		\item Für jeden Vektor muss eingetragen werden, welches die Anfangsadresse der Interrupt Service Routine (ISR) ist, d.h. die IVT ist nichts anderes als eine Tabelle(Array) von Funktionspointern.
	\end{itemize}
\end{itemize}
\subsection{Model View Controller (MVC) \buch{p.11}}
\begin{itemize}
	\item Beim MVC wirkt das Model als Server, die Views sind Clients
	\item Jeder Client meldet beim Server an, welche Ereignisse ihn interessieren
	\item Diese Anmeldung, bzw. Registrierung erfolgt über eine Funktion, welche der Server anbietet:
	\begin{lstlisting}[style=C]
int Foo_registerCb(Foo_Event e,Foo_CbFunction f);
// registers function 'f' on event 'e'
	\end{lstlisting}
	\item Der Server trägt diesen Funktionspointer in eine Tabelle ein und ruft bei Eintreten des Ereignisses alle registrierten Funktionen auf.
	\item[+] Die Views werden asynchron genau informiert, wenn sich etwas geändert hat im Model
	\item[+] An und für sich sind alle registrierten Funktionen nichts anderes als Eventhanler eines bestimmten Events
	\item[+] Dadurch wird die Darstellung(die Definition der registrierten Funktionen) sauber von den Daten (Model) entkoppelt 
\end{itemize}
