\section{Style guide und Language Linkage}
\subsection{Style guide}
\begin{itemize}
	\item Programme werden für Progammierer geschrieben, nicht für Compiler
	\item Unwichtiges nicht überbewerten
	\item Seien Sie konsistent und konsequent!
\end{itemize}


\subsection{Language Linkage}
  \textbf{Plain Old Datatypes (POD):} dazu gehören char, short, int, long, jeweils signed und unsigned, float und double.
  
  Beim Übersetzten in Assembler müssen eindeutige Labels entstehen. In C ist dies kein Problem,
  da die Funktionsnamen bereits eindeutig sind. Hingegen in C++ ist dies nicht der Fall. Deshalb weist der
  Compiler den Funktionen einen eindeutigen Namen zu (dieser Prozess wird als Mangling bezeichnet).\\
  
  Beispiele für Mangling:\\
  \begin{tabular}{|l|lll|}
  \hline
    C   & foo() & $\rightarrow$ & \_foo \\
  \hline
    C++ & foo(int) & $\rightarrow$ & \_foo\_i \\
        & foo(double, int) & $\rightarrow$ & \_foo\_d\_i \\
        & MyClass::foo(int) & $\rightarrow$ & \_MyClass\_foo\_i \\
  \hline
  \end{tabular}
  
  Wird in C++ eine Funktion foo(int) aufgerufen welche in einer compilierten C-Library vorliegt, sucht der
  Compiler nach einer Funktion \_foo\_i. Diese exisiert aber nicht da sie \_foo heisst. Deshalb muss dem Compiler
  mitgeteilt werden, dass es sich um eine C-Funktion handelt.\\
  
\begin{lstlisting}[style=Cpp]
extern "C" void foo(int);    \\ use C language linkage
extern void goo(int);        \\use C++ language linkage
void hoo(int);               \\use C++ language linkage
extern "C++" void koo(int);  \\use C++ language linkage

//optimale Anwendung von extern "C"
#ifdef __cplusplus
extern "C"
{
#endif
// list multiple prototypes with C linkage, or
// #include C header file(s)
#ifdef __cplusplus
}
#endif
\end{lstlisting}
